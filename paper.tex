\documentclass[sigconf]{acmart}
%\documentclass[sigconf,review,anonymous]{acmart}

\usepackage{enumitem}
\setlist{leftmargin=1.5em}
\setlength\itemsep{0.5em}

\setcopyright{acmlicensed}
\copyrightyear{2018}
\acmYear{2018}
\acmDOI{XXXXXXX.XXXXXXX}

\acmConference[Conference acronym 'XX]{Make sure to enter the correct
  conference title from your rights confirmation emai}{June 03--05,
  2018}{Woodstock, NY}
\acmISBN{978-1-4503-XXXX-X/18/06}

\begin{document}
\title[Denicek: Programming Substrate for Concrete, Collaborative, Interactive
  Programming]{{\scshape Denicek}: Programming Substrate for Concrete, \\Collaborative, Interactive Programming}

\author{Tomas Petricek}
\email{tomas@tomasp.net}
\orcid{0000-0002-7242-2208}
\affiliation{%
  \institution{Faculty of Mathematics and Physics, Charles University}
  \city{Prague}
  \country{Czech Republic}
}

\begin{abstract}
Research on interactive programming systems gave rise to a range of programming experiences,
including programming by demonstration, local-first collaborative editing, structure
editing, schema and code co-evolution, provenance tracking and output invalidation. Those
experiences are compelling, but they are hard to implement on the basis of existing programming
languages and systems.

We contribute the Denicek computational substrate. Denicek represents a program as a series of edits
that construct or transform a document consisting of data and formulas. Denicek provides two
primitive operations on series of edits, merging and conflict checking, that form the backbone
of the implementation of the aforementioned programming experiences.

We discuss the architecture of Denicek, document notable design considerations and elaborate
the implementation of the programming experiences listed above. To evaluate the proposed
architecture, we use Denicek as the basis of a simple innovative data exploration environment.
The case study shows that the Denicek computational substrate provides a pathway to the design
of richer and more accessible interactive programming systems.
\end{abstract}

\keywords{Do, Not, Us, This, Code, Put, the, Correct, Terms, for,
  Your, Paper}

%\begin{teaserfigure}
%  \includegraphics[width=\textwidth]{sampleteaser}
%  \caption{Seattle Mariners at Spring Training, 2010.}
%  \Description{Enjoying the baseball game from the third-base
%  seats. Ichiro Suzuki preparing to bat.}
%  \label{fig:teaser}
%\end{teaserfigure}
\maketitle

\newpage
~
\newpage
\section{Introduction}
intro

\subsection{Programming Experiences}
list

\subsection{Substrate}
how it works very roughly

\subsection{Contributions}
one main thing - substrate - with other things


\section{Background}
all the references

\section{Walkthrough}
conference organizer

1) PBD and interaction

2) Merging and schema evolution

3) Evaluation and maybe provenance

\section{Architecture}

also NAIVE REALISM

\section{Implementation}
(how each of the features is achieved)


1) PBD (create textbox but not button)

2) Merging (refactoring)

3) Interaction (add button)

4) Evaluation and invalidation

5) Schema evolution

\section{Design Discussion}
\section{Case study}
(data science environment)

\section{Discussion}
\subsection{Heuristic evaluation}
\subsection{Limitations}
maybe


DESIGN PROCESS = formative examples + evaluation case study

xx
\newpage
xx

~\\
introduction \\
background \\
related work \\
design process / goals \\
case study \\
formative research \\
formative study \\
analysis \\
system \\
implementation \\
evaluation / heuristic evaluation \\
discussion and limitations \\

\newpage

\section{Introduction}

The computational substrate using which software is built determines the capabilities that the
software can provide. An imperative substrate that views programs as instructions modifying
bytes in memory makes it almost impossible to allow end-user inspection or reprogramming of
running software.

A computational substrate defines what software is built from. This may be objects as in
Smalltalk, lists as in Lisp, or memory with data and code as in UNIX/C.
The different substrates enable different kinds of programming experiences.
For example, object-oriented programming has historically been linked to the development
of graphical user interfaces (where objects can correspond to elements on the screen).
It has also enabled the development of visual programming environments such as the Alternate
Reality Kit, based on message sending between objects.

In principle, any computational substrate can be used to develop any programming experience,
but the greater the impedance mismatch between the substrate and the desired experience,
the more difficult it will be to provide the experience and combine it with the rest of the
system and other programming experiences developed for the system. (One can implement support
for programming-by-demonstration using C/C++, for example as part of a game scripting engine,
but it will not work with the rest of the ordinary C/C++ ecosystem.)

\subsection{Substrate}

The question asked in this paper is, what would be the ideal programming substrate
for supporting a range of programming experiences that make programs more
collaborative, transparent and allows for a gradual transition from non-programmer
to a programmer. We want a programming substrate that makes it easy to develop
programming experiences such as:

\begin{itemize}
\item \emph{Programming by demonstration} --
  Allow non-programmers to construct simple programs by performing examples of the expected behaviour. \cite{leiva-2021-rapido}.
\item \emph{Local-first collaboration} --
  Multiple users should be able to use and modify a single program, preferrably without requiring a central server. \cite{kleppmann-2019-local}
\item \emph{Provenance tracking} --
  The execution of the program should leave an understandable trace that lets the user understand why program resulted in a particular result.
\item \emph{Schema evolution [extra-ish]} --
  When the user evolves the structure of the program, data and code should co-evolve automatically to match the new structure.
\item \emph{Notational freedom [extra-ish?]} --
  Allow users to adapt the program using a notation that suits them and is appropriate for the programming task at hand. [Joel]
\item \emph{Concrete programming [extra?]} --
  It should be possible to reuse parts of program or program logic without constructing abstractions, for example by managed copy \& paste.\cite{edwards-2006-copypaste,edwards-2022-copypaste}
\end{itemize}

substrate as defined by \cite{jakubovic-2022-ladder}

% pbd
\cite{leiva-2021-rapido,cypher-1993-pbd}
\cite{chen-2023-miwa}

% lcoal first
\cite{kleppmann-2019-local,klokmose-2024-mywebstrates}
% provenance
\cite{ko-2004-whyline,ko-2009-whyline,krebs-2023-probelog}
\cite{ricciotti-2017-imperative,perera-2012-functional}
\cite{perera-2022-linked}


% visualizations of results or execution
% https://www.dcs.warwick.ac.uk/pvw04/p01.pdf
% https://dl.acm.org/doi/10.1145/3313831.3376494

% previews - livelits

% Schema evolution
% our paper; see https://openproceedings.org/2023/conf/edbt/paper-160.pdf - good refs in 2.2

% https://x.com/jonathoda/status/1185888711210389504

% also cool: https://arxiv.org/pdf/2303.06777

% Hazel family of things
% structure editing - sandblocks

Joel's definition of substrate in Onward!
Bret Victor talk
\url{https://www.youtube.com/watch?v=ef2jpjTEB5U}

In what ways is a substrate "natural"?

thinglab - create line by cloning, it sticks to mouse pointer, clicking sticks it to something else
squeak - has all the browsers (method search...)


computational substrate
how it differs from computational media?
more low-level - media suggests that there it comes

\section{The whatever system}
\subsection{Document + Edits}
defines
\begin{itemize}
  \item selectors
  \item nodes
  \item edits
\end{itemize}

\subsection{Walkthrough}
* todo list? (or counter, but that is a bit boring)

\section{Themes}
* programming by demonstration
  - binding interactions to gui elements (event handlers)
* provenance tracking
  - Amy Ko's whyline, Probe Log by HPI, enables linked visualizations
* merging of edit histories / collaborative editing
  - bonus - can share restricted link to allow users fill out
    forms (allow partial edits only / def by selector?)
* scehma change - change data \& code accordingly
* everything is an edit
  - interaction with the GUI
  - evaluation? tbd
* copy \& paste abstraction
    (requires finishing new approach to formulas!)
  - edit before copy to propagate edit to other places
    (or edit after copy to make it specific to a case)
  - higher order copying from https://tomasp.net/academic/papers/copy-paste/paint22.pdf
* augmenters
  - cf. bonnie nardi (calls them something else - Jonathan says)
  - add programming by demonstration data wrangling gui to table (trigger interactions)
    cf. lorgnette

\section{Applications}
* todo list / counter / maybe too simple
* (if used in the walkthrough, maybe something else? board game as in varv - tic tac toe? or 7guis?)
* conference organizer
* data exploration (ala histogram)
* linked charts

\section{Extras}
* metablocks?
* self-sustainability
* some non-browser implementation of this (as in Varv?)

explicit structure
self-sustainability
notational freedom

\newpage
~

Maybe have 'enabled' for edits afterall?
(we can merge with conflicts and disable some edits, but keep them in history for info)

NOTES
type Edit =
  { Kind : EditKind
    Dependencies : Selectors list }  -- only needed for evaluated edits

VALUE vs STRUCTURE distinction
* good in theory, nice for implementation
* tricky to use! needs some assistance tools

TODO - things to work on
* "represent" edits somewhere in document as "library of functions"
  and then call those from buttons (rather than embedding them directly)
  allow some kind of abstraction (as in Histogram) to make them reusable
* figure out how to do evaluation better
  (based on the stored abstractions? but need to store provenance...)

SEMANTIC CONDITIONS
\url{https://www.youtube.com/watch?v=NBnc2ToS_j0}
(has a section on this in background)

SUBSTRATE DESIGN PROBLEMS
* selectors - all for structure / index for data
  (but it is useful to allow others...)
  (multiselect also bad for checks!)
* groups/conditions/preconditions
  (c.f. email to jonathan)
  tried conditions on edits; trying groups with check edits
* what to do with "disabled edits"? for example when we remove all checked
  (before, this created edit groups with "check" but if the check was false,
  the group was ignored and this messed up merging - because we wouldn't know if the
  edit had any effect or not)

Evaluation
* evaluated edits have to be migrated to the end
  (if there are conflicts, they are dropped)
  Think of this as maintaining a tree:

  e3
  |
  e2   evaluated
  |  /
  e1
  |
  e0

  this has to be serialized as e0 -> e1 -> e2 -> e3 -> evaluated

  evaluated edits do not became part of the main history
  but hang on the side

ISSUES
* if we merge a thing with saved-interactions with something, hashes will change!

NOTES
* ListAppendFrom - we need this, because we cannot encode this.
* for records, we can RecordAdd(sel, fld, ..) @ Copy(sel @ [Field fld], src) but
  this does not work for lists - because we do not know the index!
  (and we cannot look into current document, because it will differ for saved-interactions)


TODO
* many things with <tag> selectors currently do not work
  (e.g. `matches` for highlighting) because if we collect path of a current node,
  we collect indices and get /some/2/another - and cannot tell if this matches
  /some/<li>/another - we'd have to collect more detailed path info!

INTERACTION
* replay stored event handlers against the old version?
  (this way, adding an item to a speakers list gets migrated \& adds a new table row!)
* simlarly!! we need merge in order to apply edits to multiple targets
  (when you remove all items in a list, the indices change)
  (but I guess we should do this against version at the time of saving too....)

[this \& evaluation = the unreasonable effectiveness of merging]

Notes on storing and reusing edits
* references need to be represented as references so that they get updated
  (NO! not if we reply them against old version, which seems better - but there are 2 design choices)
* how to apply them to multiple targets? use Move to update the selectors instead of
  replacing the prefix manually

IDEA: Type check edit groups to ensure they preserve structure but not individual edits eg when adding list item

CONDITIONALS
https://toby.li/files/p311-radensky.pdf


REMAINING IMPLEMENTATION TODOs:

* Some kind of provenance visualization
* Some kind of matchers/transformers mechanism (ideally to add interactive buttons to tables)
* Apply to all (remove completed in TODO)

\newpage
~

\bibliographystyle{plain}
\bibliography{paper}

\end{document}
